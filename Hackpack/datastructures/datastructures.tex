\documentclass[../hackpack.tex]{subfiles}

\begin{document}

\section{Data Structures}

\subsection{UFDS}
\textit{Construction: $O(n)$} 
\\ 
\textit{Union Find with Path Compression: $\approx O(1)$}
\\
Union Find Data Structure, or Disjoint Set, uses an array \textit{\textbf{id}} to keep track of the root of the set of each element.
The following implementation is \textbf{0-indexed}.
\lstinputlisting{ufds.txt}

\subsection{Segment Tree}
\textit{Construction: $O(n)$}
\\
\textit{Query: $O(\log{n})$}
\\
Segment Tree is a data structure that can be used to quickly answer mutliple range queries on an array. It works by creating a binary tree, where each node 
represents a range of the array. The root node represents the full array, its children represent the left and right halves, and so on.
\lstinputlisting[caption=
https://cp-algorithms.com/data\_structures/segment\_tree.html \\\hspace{\textwidth}
https://www.hackerearth.com/practice/data-structures/advanced-data-structures/segment-trees/tutorial/
]
{segment_tree.txt}
% https://cp-algorithms.com/data_structures/segment_tree.html

\end{document}